\section{Appendix: Algorithms}
\label{sec:appendix_algorithms}  

\subsection{Positions Accuracy Ellipses}
\label{sec:algo_position_accuracy_ellipses} 

According to the \href{https://en.wikipedia.org/wiki/68%E2%80%9395%E2%80%9399.7_rule}{68–95–99.7} rule, 68.27\%, 95.45\% and 99.73\% of the values lie within one, two and three standard deviations of the mean, respectively. \\

Therefore, 95\% is equivalent to two standard deviations ($\sigma$). \\

Please note that currently only the Cartesian variables are used, and are assumed to be valid for the target reports position (and north direction), not for a system center. \\

The usage WGS-84 variables will be added in one of the next releases.

\subsection{ADS-B}

The accuracy values where taken from \href{https://mode-s.org/decode/adsb/uncertainty.html}{The 1090MHz Riddle} online book.

\begin{center}
 \begin{table}[H]
  \begin{tabularx}{\textwidth}{ | X | l | l | }
    \hline
    \textbf{ASTERIX Data Item} & \textbf{Variable} & \textbf{Comment} \\ \hline
    210.VN & mops\_version & Used to distinguish between DO-260 versions \\ \hline
    090.NACp & nac\_p  & Used for V1 \& V2 transponders \\ \hline
    090.NUCp or NIC & nucp\_nic  & Used for V0 transponders \\ \hline
  \end{tabularx}
  \caption{ADS-B Position Accuracy Variables}
\end{table}
\end{center}

\subsubsection{V0 Transponders}

\begin{table}[H]
  \center
  \begin{tabular}{ | l | l | l | l |}
    \hline
    \textbf{NUCp} & \textbf{HPL} & \textbf{RCu} & \textbf{$\sigma$} \\ \hline
    9   & < 7.5 m            & < 3 m             & 1.5  \\ \hline
    8   & < 25 m             & < 10 m            & 5  \\ \hline
    7   & < 0.1 NM (185 m)   & < 0.05 NM (93 m)  & 46.5  \\ \hline
    6   & < 0.2 NM (370 m)   & < 0.1 NM (185 m)  & 92.5  \\ \hline
    5   & < 0.5 NM (926 m)   & < 0.25 NM (463 m) & 231.5  \\ \hline
    4   & < 1 NM (1852 m)    & < 0.5 NM (926 m)  & 463  \\ \hline
    3   & < 2 NM (3704 m)    & < 1 NM (1852 m)   & 926  \\ \hline
    2   & < 10 NM (18520 m)  & < 5 NM (9260 m)   & 4630  \\ \hline
    1   & < 20 NM (37040 m)  & < 10 NM (18520 m) & 9260  \\ \hline
    0   & > 20 NM (37040 m)  & > 10 NM (18520 m) & -  \\ \hline
  \end{tabular}
  \caption{NUCp Values}
\end{table}

\begin{itemize}
\item NUCp: Navigation Uncertainty Category - Position
\item HPL: Horizontal Protection Limit
\item RCu: 95\% Containment Radius - Horizontal
\item $\sigma$: Standard deviation in meters, from RCu/2
\end{itemize}
\ \\

For the NUCp values, the taken for display are listed in the $\sigma$ column, resulting in error circles (same value for both coordinates). Please note that for the NUCp value 0, no error ellipse is displayed, since the accuracy value is unkown.

\subsubsection{V1 \& V2 Transponders}

\begin{table}[H]
  \center
  \begin{tabular}{ | l | l | l | l |}
    \hline
    \textbf{NACp} & \textbf{EPU (HFOM)} & \textbf{VEPU (VFOM)} & \textbf{$\sigma$} \\ \hline
    11  & < 3 m              & < 4 m  & 1.5 \\ \hline
    10  & < 10 m             & < 15 m & 5 \\ \hline
    9   & < 30 m             & < 45 m & 15 \\ \hline
    8   & < 0.05 NM (93 m)   & & 46.5 \\ \hline
    7   & < 0.1 NM (185 m)   & & 92.5 \\ \hline
    6   & < 0.3 NM (556 m)   & & 278 \\ \hline
    5   & < 0.5 NM (926 m)   & & 463 \\ \hline
    4   & < 1.0 NM (1852 m)  & & 926 \\ \hline
    3   & < 2 NM (3704 m)    & & 1852 \\ \hline
    2   & < 4 NM (7408 m)    & & 3704 \\ \hline
    1   & < 10 NM (18520 km) & & 9260 \\ \hline
    0   & > 10 NM or Unknown & & - \\ \hline
  \end{tabular}
  \caption{NACp Values}
\end{table}

\begin{itemize}
\item NACp: Navigation Accuracy Category - Position
\item EPU (HFOM): 95\% horizontal accuracy bounds, Estimated Position Uncertainty (EPU) a.k.a. Horizontal Figure of Merit (HFOM)
\item VEPU (VFOM): 95\% vertical accuracy bounds, Vertical Estimated Position Uncertainty (VEPU) a.k.a. Vertical Figure of Merit (VFOM)
\item $\sigma$: Standard deviation in meters, from EPU/2
\end{itemize}
\ \\

For the NACp values, the taken for display are listed in the $\sigma$ column, resulting in error circles (same value for both coordinates). Please note that for the NACp value 0, no error ellipse is displayed, since the accuracy value is unkown.

\subsection{MLAT}

The description of the accuracy values where taken from \textit{EUROCONTROL Specification for Surveillance Data Exchange ASTERIX Part 14 Category 020 Multilateration Target Reports Appendix A: Reserved Expansion Field} (EUROCONTROL-SPEC-0149-14A) document.

\begin{center}
 \begin{table}[H]
  \begin{tabularx}{\textwidth}{ | X | l | l | }
    \hline
    \textbf{ASTERIX Data Item} & \textbf{Variable} & \textbf{Comment} \\ \hline
     REF.PA.SDW.SDW (Latitude Component) & pos\_std\_dev\_lat\_deg &  \\ \hline
     REF.PA.SDW.SDW (Longitude Component) & pos\_std\_dev\_long\_deg  &  \\ \hline
     REF.PA.SDW.COV-WGS (Lat/Long Covariance Component) & pos\_std\_dev\_latlong\_corr\_coeff  &  \\ \hline
     REF.PA.SDC.SDC (X-Component) & pos\_std\_dev\_x\_m &  \\ \hline
     REF.PA.SDC.SDC (Y-Component) & pos\_std\_dev\_y\_m  &  \\ \hline
     REF.PA.SDC.COV-XY (Covariance Component) & pos\_std\_dev\_correlation\_coeff  &  \\ \hline
\end{tabularx}
  \caption{MLAT Position Accuracy Variables}
\end{table}
\end{center}

\begin{itemize}
\item SDC.SDC values: Standard Deviation of Position of the target expressed in Cartesian coordinates, in meters
\item SDC.COV-XY: XY Covariance Component, the unit is listed as in meters, but should be in $m^2$ and is asssumed as such
\item SDW.SDW values: Standard Deviation of Position of the target expressed in WGS-84, in degrees
\item SDW.COV-WGS: Lat/Long Covariance Component, the unit is listed as in degrees, but should be in $deg^2$ and is asssumed as such
\end{itemize}
\ \\

Notes:
\begin{itemize}
\item XY covariance component = sign {Cov(X,Y)} * sqrt {abs [Cov (X,Y)]}
\item WGS-84 covariance component = sign {Cov(Lat,Long)} * sqrt {abs [Cov (Lat,Long)]}
\end{itemize}

\subsection{Radar}

For each Radar plot, the DBOVariables 'ds\_id' and 'detection\_type' are used to define from which data source the plot originated, and what type of plot was measured. \\

From the data source information (see \nameref{sec:task_manage_datasources_table_content}) the radar standard deviations are collected. If no data source specific values are set, the default values (see \nameref{sec:others_radar_default_accuracies}) are used. \\

Based on the plot type defined by 'detection\_type', one (single technology) or the minimum/maximum of the standard deviations (for combined plots, defined by 'Use Radar Minimum StdDev' flag) is used. If the 'detection\_type' is not set or is 0 (no detection), the PSR values are used. \\ 

The standard deviates used for display purposes are calculated based on the range standard deviation and the azimuth standard deviation multiplied by the circumference at the given range, and are of course rotated by the measurement azimuth.

\subsection{RefTraj}

The usage WGS-84 variables will be added in one of the next releases.

\subsection{Tracker}

The description of the accuracy values where taken from \textit{EUROCONTROL Specification for Surveillance Data Exchange ASTERIX Part 9 Category 062 SDPS Track Messages} (EUROCONTROL-SPEC-0149-9) document.

\begin{center}
 \begin{table}[H]
  \begin{tabularx}{\textwidth}{ | X | l | l | }
    \hline
    \textbf{ASTERIX Data Item} & \textbf{Variable} & \textbf{Comment} \\ \hline
     500.APW.APW (Latitude Component) & pos\_std\_dev\_lat\_deg &  \\ \hline
     500.APW.APW (Longitude Component) & pos\_std\_dev\_long\_deg  &  \\ \hline
     500.APC.APC (X-Component) & pos\_std\_dev\_x\_m &  \\ \hline
     500.APC.APC (Y-Component) & pos\_std\_dev\_y\_m  &  \\ \hline
     500.COV.COV (XY Covariance Component) & pos\_std\_dev\_xy\_corr\_coeff  &  \\ \hline
\end{tabularx}
  \caption{Tracker Position Accuracy Variables}
\end{table}
\end{center}

\begin{itemize}
\item APC.APC values: Estimated accuracy (i.e. standard deviation) of the calculated position of a target expressed in Cartesian co-ordinates, in meters
\item COV.COV: XY Covariance Component, the unit is listed as in meters, but should be in $m^2$ and is asssumed as such
\item APW.APW values: Standard Deviation of Position of the target expressed in WGS-84, in degrees
\end{itemize}
\ \\

Notes:
\begin{itemize}
\item XY covariance component = sign {Cov(X,Y)} * sqrt {abs [Cov (X,Y)]}
\end{itemize}
